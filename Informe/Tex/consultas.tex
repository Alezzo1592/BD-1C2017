\section{Implementación  de Funcionalidades}

Una vez que terminado el modelo del problema, el siguiente paso fue comenzar a implementar la base de datos. Es decir generar en mySQL las distintas tablas, entradas y consultas necesarias para resolver el problema.\\

A continuación, mostraremos como se implementaron cada uno de los Store Procedures correspondientes a las consultas necesarias para cumplir con los requerimientos del punto 2 del enunciado:\\

\begin{itemize}
\item{\textbf{Listado de inscritos en cada categoría.}\\
select nombre from Competidor comp inner join 
(select nroCertificadoITFCompetidor 
from ParticipaIndividualmente p 
inner join (select idCompetencia from (select idCompetencia, nombreModalidad from Competencia c
inner join Modalidad m on c.idModalidad = m.idModalidad) t1 where t1.nombreModalidad = 'mod1') t2
on p.idCompetencia = t2.idCompetencia) t3 
on comp.nroCertificadoITF = t3.nroCertificadoITFCompetidor;

}

\item País que obtuvo mayor cantidad de medallas de oro, plata y bronce.
\item Sabiendo que las medallas de oro van 3 puntos, las de plata 2 y las de bronce 1 punto, se quiere realizar un ranking de puntaje por país y otro por escuela.
\item Dado un competidor obtener la lista de categorías donde haya participado y el resultado obtenido.

\item{ \textbf{Medallero por escuela.}\\
CREATE PROCEDURE `medalleroPorEscuela`()\\
BEGIN\\
select E.nombre,\\
	count(case when PC.posicion = 1 then 1 else NULL end) + count(case when PE.posicion = 1 then 1 else NULL end) as oro, \\
	count(case when PC.posicion = 2 then 1 else NULL end) + count(case when PE.posicion = 2 then 1 else NULL end) as plata,  \\
    count(case when PC.posicion = 3 then 1 else NULL end) + count(case when PE.posicion = 3 then 1 else NULL end) as bronce\\
    from Escuela E\\
		join Competidor C on C.idEscuela = E.idEscuela\\
		left join Equipo Q on C.idEquipo = Q.idEquipo\\
		left join PodioPorCompetidor PC on C.nroCertificadoITF = PC.nroCertificadoITF\\
		left join podioPorEquipo PE on PE.idEquipo = Q.idEquipo\\
		
		group by E.idEscuela\\
		order by oro desc, plata desc, bronce desc\;\\
END\\
}

\item{\textbf{Listado de árbitros por país.}\\
CREATE PROCEDURE `arbitroPorPais`(in nombrePais varchar(45))\\
BEGIN\\
SELECT A.* FROM Arbitro A\\
	JOIN Pais P on A.idPais = P.idPais\\
    WHERE nombrePais = P.nombre;\\
END\\
}

\item{\textbf{La lista de todos los árbitros que actuaron como arbitro central en las modalidades de combate.}\\
CREATE PROCEDURE `arbitrosCentrales`(in modalidad VARCHAR(45))\\
BEGIN\\
select A.* from Arbitro A\\
	join Rol R on R.idRol = A.idRol\\
	join Arbitraron AR on A.idArbitro = AR.idArbitro\\
    join Competencia C on AR.idCompetencia = C.idCompetencia\\
    join Modalidad M on M.idModalidad = C.idModalidad\\
    where M.nombreModalidad = modalidad and R.descripcion = 'central'\;\\
END\\
}

\item{\textbf{La lista de equipos por país.}\\
CREATE PROCEDURE `equiposPorPais`(in nombrePais varchar(45))\\
BEGIN\\
SELECT E.nombre FROM equipos E\\
	JOIN Competidor C ON C.idEquipo = E.idEquipo\\
	JOIN Escuela S ON C.idEscuela = S.idEscuela\\
	WHERE nombrePais = P.nombre\\
END\\
}
\end{itemize}