\section{Modelo Relacional}

Una vez obtenido el diseño de la base de datos que deseamos modelar, el siguiente paso es comenzar a transformar dicho modelo en un esquema de datos lógico. Para ello utilizamos el Modelo Relacional, en el que cual reduciremos las distintas Entidades, con sus atributos e interrelaciones, a una estructura llamada esquema de relación. Esta estructura estará compuesta por un nombre, que se obtendra apartir de 

\begin{itemize}

\item{Escuela(idEscuela,Nombre)
PK\=CK\=\{idEscuela\}}
%Me parece que podriamos sacar la escuela y ponersela como atributo al maestro

\item{Maestro(idMaestro,Apellido,Nombre,Graduacion,idEscuela,Pais,NumPlaca)
PK\=\{idMaestro\}
FK\=\{idEscuela\}
}

\item{Coach(idCoach,Apellido,Nombre,Graduacion,NroCertificadoITF,Foto,idEscuela,DNI)
PK\=\{idCoach\}
FK\=\{DNI\}
}

\item{Competidor(Apellido,Nombre,DNI,fechaDeNac,Genero,Graduacion,NroCertificadoITF,Peso,Foto,idEscuela,idEquipo)
PK\=\{DNI\}
FK\=\{idEscuela,idEquipo\}
}
\item{Equipo(idEquipo,Nombre)
PK\=\{idEquipo\}
}

\item ParticipaEn()

\item{EquipoParticipaEn()}

\item Modalidad()

\item Competencia(idComp,idEquipo1puesto,idEquipo2puesto,idEquipo3puesto)

\item CompetenciaPorEquipo(idCompEquipo)

\item Llave(idLlave,idRing,idComp,idCompEquipo)

\item LlavePorEquipos()

\item Ring(idRing)

\item Arbitro(idArbitro,Apellido,Nombre,Graduacion,idRol,idRing)

\item Rol(idRol,Descripcion)




\end{itemize}

