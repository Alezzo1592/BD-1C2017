\section{Modelo Relacional}

Una vez obtenido el diseño de la base de datos que deseamos modelar, el siguiente paso es comenzar a transformar dicho modelo en un esquema de datos lógico. Para ello utilizamos el Modelo Relacional, en el que cual reduciremos las distintas Entidades, con sus atributos e interrelaciones, a una estructura llamada esquema de relación. Esta estructura estará compuesta por un nombre, que se obtendra apartir de 

\begin{itemize}

\item{Escuela(\underline{idEscuela}, nombre,idPais)\\
PK\=CK\=\{idEscuela\}}
%Me parece que podriamos sacar la escuela y ponersela como atributo al maestro

\item{Maestro(\underline{idMaestro}, apellido, nombre, graduacion, \dotuline{idEscuela}, numPlaca)\\
PK\=\{idMaestro\} \\
FK\=\{idEscuela\} \\
}

Pais(\underline{idPais}, nombre)

\item{Coach(\underline{nroCertificadoITF}, apellido, nombre, graduacion , idFoto, \dotuline{idEscuela}, \dotuline{DNI})\\
PK\=\{nroCertificadoITF\} \\
FK\=\{DNI\} \\
}

\item{Competidor(\underline{nroCertificadoITF}, apellido, nombre, fechaDeNac, genero, graduacion, dni, peso, idFoto, \dotuline{idEscuela}, \dotuline{idEquipo}) \\
PK\=\{nroCertificadoITF\} \\
FK\=\{idEscuela,idEquipo\}
}
\item{Equipo(\underline{idEquipo},nombre)\\
PK\=\{idEquipo\} \\
}

PodioPorCompetidor(idCompetencia, nroCertificadoITF,posicion)
%idcompetencia y nroCertificado son PK y FK

\item ParticipaEn()

\item EquipoParticipaEn()

\item Modalidad()

\item{GrupoArbitro(\dotuline{idGrupoArbitro}, \dotuline{idCompetencia})\\
}

\item Competencia(\underline{idCompetencia},  \dotuline{idModalidad})

\item Ring(\underline{idRing}, \dotuline{idCompetencia})

\item Arbitro(\underline{idArbitro}, apellido, nombre, graduacion, \dotuline{idRol} ,\dotuline{idGrupoArbitro})\\

\item Rol(\underline{idRol}, descripcion)




\end{itemize}

