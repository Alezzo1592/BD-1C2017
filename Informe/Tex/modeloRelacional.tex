\newpage
\section{Modelo Relacional}

Una vez obtenido el diseño de la base de datos que deseamos modelar, el siguiente paso es comenzar a transformar dicho modelo en un esquema de datos lógico. Para ello utilizamos el Modelo Relacional, en el que cual reduciremos las distintas Entidades, con sus atributos e interrelaciones, a una estructura llamada esquema de relación. Esta estructura estará compuesta por un nombre y una lista de atributos, que se obtendrán a partir de las entidades y sus relaciones.

\begin{itemize}

\item{Pais(\underline{idPais}, nombre)\\
PK\=CK\=\{idPais\}}

\item{Escuela(\underline{idEscuela}, nombre, \dashuline{idPais})\\
PK\=CK\=\{idEscuela\}\\
FK\=\{idPais\}
}

\item{Maestro(\underline{numPlaca}, apellido, nombre, graduacion, \dashuline{idEscuela})\\
PK\=\{numPlaca\} \\
FK\=\{idEscuela\} \\
}

\item{Coach(\underline{nroCertificadoITF}, apellido, nombre, graduacion, \dashuline{dni}, idFoto, \dashuline{idEscuela})\\
PK\=\{nroCertificadoITF\} \\
CK\=\{nroCertificadoITF,dni	\} \\
FK\=\{idEscuela,idFoto\} \\
}

\item{Equipo(\underline{idEquipo}, nombre, generoEquipo)\\
PK\=\{idEquipo\} \\
}

\item{Competidor(\underline{nroCertificadoITF}, apellido, nombre, fechaNacimiento, genero, graduacion, \dashuline{idEscuela}, dni, peso, idFoto, \dashuline{idEquipo}, esTitular) \\
PK\=\{nroCertificadoITF\} \\
CK\=\{nroCertificadoITF,dni\} \\
FK\=\{idEscuela,idEquipo\}
}

\item{Modalidad(\underline{idModalidad}, edadMinima, edadMaxima, sexo, pesoMinimo, pesoMaximo, nombreModalidad, graduacion) \\
PK\=\{idModalidad\}
}

\item{Competencia(\underline{idCompetencia}, \dashuline{idModalidad})
PK\=\{idCompetencia\}\\
FK\=\{idModalidad\}
}

\item{Rol(\underline{idRol}, descripcion)
PK\=CK\=\{idRol\}
}

\item{Arbitro(\underline{idArbitro}, apellido, nombre, graduacion, \dashuline{idRol}, \dashuline{idPais}, idGrupoArbitro)\\
PK\=\{idArbitro\}\\
FK\=\{idRol\}
}

\item{Arbitraron(numeroRing, \underline{idCompetencia}, \underline{idArbitro)}
PK\= \{(idCompetencia,idArbitro)\}\\
FK\=\{idCompetencia,idArbitro\}
}

\item{PodioPorCompetidor(\underline{\dashuline{idCompetencia}}, \underline{\dashuline{nroCertificadoITF}}, posicion)
PK\= \{(idCompetencia,nroCertificadoITF)\}\\
FK\=\{idCompetencia,nroCertificadoITF\}
}

\item{PodioPorEquipo(\underline{\dashuline{idCompetencia}}, \underline{\dashuline{idEquipo}}, posicion)
PK\= \{(idcompetencia,idEquipo)\}\\
FK\=\{idCompetencia,idEquipo\}
}

\item{ParticipaIndividualmente(\underline{\dashuline{idCompetencia}}, \underline{\dashuline{nroCertificadoITFCoach}}, \dashuline{nroCertificadoITFCompetidor})\\
PK\=CK\= \{(idCompetencia,nroCertificadoITFCompetidor)\}\\
FK\=\{nroCertificadoITFCoach,nroCertificadoITFCompetidor,idCompetencia\}
}

\item{ParticipaEquipo(\dashuline{idCompetencia}, \dashuline{nroCertificadoITFCoach}, \underline{\dashuline{idEquipo}})\\
PK\= \{(idEquipo,nroCertificadoITFCoach)\}\\
CK\= \{(idEquipo,nroCertificadoITFCoach),(idEquipo,idCompetencia)\}\\
FK\=\{nroCertificadoITFCoach,idEquipo,idCompetencia\}
}

\end{itemize}

\subsection{Restricciones}

%Las restriccionas estan escritas informalmente. Habria que corregirlas con los nombres de los atributos de cada entidad.

A continuación, describiremos las restricciones que se deben mantener a la hora de generar las distintas entidades y relaciones para así modelar correctamente el problema.

\begin{itemize}
\item Al asignar un competidor a un equipo se debe chequear que sea del mismo sexo que el resto de sus compañeros de equipo y que pertenezcan a la misma escuela.
\item Los Coachs que también participan como competidores tienen datos repetidos. Estos datos deben ser consistentes.
\item Al crear una Competencia se deben tener asignados a todos los árbitros necesarios para el desarrollo de la misma. Es decir, se le debe asignar el rol correspondiente a cada uno de los árbitros; y se deberá controlar que no haya mas de un arbitro para los roles de presidente de mesa y arbitro central.
\item Los competidores que participan en una competencia deben tener un peso, genero, edad y graduación consistente con la modalidad de dicha competencia.
\item Los árbitros tienen la graduación suficiente para dirigir las competencias en las que participan.
\item Los Coachs deben acompañar a participantes de su escuela.
\item Hay un Coach por cada 5 alumnos inscritos en una escuela.
\end{itemize}