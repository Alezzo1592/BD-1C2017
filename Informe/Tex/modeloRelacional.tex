\section{Modelo Relacional}

Una vez obtenido el diseño de la base de datos que deseamos modelar, el siguiente paso es comenzar a transformar dicho modelo en un esquema de datos lógico. Para ello utilizamos el Modelo Relacional, en el que cual reduciremos las distintas Entidades, con sus atributos e interrelaciones, a una estructura llamada esquema de relación. Esta estructura estará compuesta por un nombre, que se obtendra apartir de 

\begin{itemize}

\item{Escuela(\underline{idEscuela}, Nombre)\\
PK\=CK\=\{idEscuela\}}
%Me parece que podriamos sacar la escuela y ponersela como atributo al maestro

\item{Maestro(\underline{idMaestro}, Apellido, Nombre, Graduacion, \dotuline{idEscuela}, Pais, NumPlaca)\\
PK\=\{idMaestro\} \\
FK\=\{idEscuela\} \\
}

\item{Coach(\underline{idCoach}, Apellido, Nombre, Graduacion, NroCertificadoITF, Foto, \dotuline{idEscuela}, \dotuline{DNI})\\
PK\=\{idCoach\} \\
FK\=\{DNI\} \\
}

\item{Competidor(Apellido, Nombre, \underline{DNI}, fechaDeNac, Genero, Graduacion, NroCertificadoITF, Peso, Foto, \dotuline{idEscuela}, \dotuline{idEquipo}) \\
PK\=\{DNI\} \\
FK\=\{idEscuela,idEquipo\}
}
\item{Equipo(\underline{idEquipo},Nombre)\\
PK\=\{idEquipo\} \\
}

\item ParticipaEn()

\item{EquipoParticipaEn()

}

\item Modalidad()

\item Competencia(\underline{idComp}, \dotuline{idEquipo1puesto}, \dotuline{idEquipo2puesto}, \dotuline{idEquipo3puesto})

\item CompetenciaPorEquipo(\underline{idCompEquipo})

\item Llave(\underline{idLlave},\dotuline{idRing},\dotuline{idComp},\dotuline{idCompEquipo})

\item LlavePorEquipos()

\item Ring(idRing)

\item Arbitro(\underline{idArbitro}, Apellido, Nombre, Graduacion, \dotuline{idRol} ,\dotuline{idRing})\\


\item Rol(\underline{idRol}, Descripcion)




\end{itemize}

